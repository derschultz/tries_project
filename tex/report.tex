\documentclass{llncs}
\usepackage{llncsdoc}

\intextsep=0.6\intextsep
\textfloatsep=0.6\textfloatsep
\abovecaptionskip=0.6\abovecaptionskip
\belowcaptionskip=0.6\belowcaptionskip

\usepackage{graphicx,amsfonts,amssymb,url,float,verbatim,listings,courier,pgfplots,tikz,tikzscale}
\usetikzlibrary{arrows}
\usetikzlibrary{positioning}
\pgfplotsset{compat=newest}

\title{Nibbles and bytes: an implementation of the trie data structure using different branching factors}

\author{Alex Schultz}

\institute{Tufts University}

\begin{document}

\maketitle

\begin{abstract}
This project consists of an implementation of the trie data structure that, instead of considering input strings character by character, considers input strings in chunks of one, two, four, or eight bits uniformly through the trie. The purpose of this was twofold: to complete a trie library that is able to work on arbitrary length input, and to do some simple testing of the behavior of the tries that use differing internal chunk sizes while working with the same data set. The C programming language was chosen for the ease it provides when working on individual bits in memory, which was necessary for this project.
\end{abstract}

\section{Introduction and Background}
\label{Introduction and Background}
%When storing data in a program, a programmer may opt to use a few different structures, depending on the type of data and other constraints. For many common use cases, it does not matter which data structure the programmer chooses to use. The main requirement that is placed upon the data structure is that it be easy to use (so that including it in the program, or allowing others to access it, does not unnecessarily burden the programmer). In addition, its performance must be consistently good under normal circumstances (that is, outside of contrived examples that cause degraded performance that may nevertheless be useful in certain cases, such as to illustrate trade-offs between different data structures). Optionally, it may also be necessary that its design fits the needs of the programmer (consider the difference between a last in, first out (LIFO) structure like a stack and a first in, first out (FIFO) structure like a queue). On the last point, the optional part comes from the fact that in some circumstances, a simple array will be fine for a program, but in others, some kind of ordering (as in a queue) or mapping (as in a hash table) may be desired. \\
When storing data in a program, a programmer may opt to use a variety of different data structures, depending on the type of data and other constraints. However, as each data structure has its own unique properties (possibly shared with other data structures), there are tradeoffs that affect the choices the programmer will make. The programmer might favor a data structure that maps more closely to the mental model they have of the program flow; on the other hand, the programmer will want to ensure that the structure does not excessively tax the system CPU or memory. Additionally, the data structure chosen must be general purpose enough to accept any and all expected input, with minimal cases where certain input would cause degraded performance. Last, the structure used might be chosen for ease of use, so that its use in the program does not overly complicate the code being written. With these requirement in mind, this project examines the trie data structure.\\

While a trie can be used to simply contain elements (in the same way a set or multiset would, but allowing in-order traversal like an ordered list would), a trie can also be used to map elements to one another, as a hash table would be used for. One of the reasons to use a trie over another data structure is that certain basic operations (search, insert, delete) can be done in time proportional to the length of the input data, rather than being done in time proportional to the amount of data already in the data structure. This is the case with some other tree data structures, such as binary search trees. For this reason, the choice to use a trie may be necessary when timing is of utmost importance. \\

A trie is a specific type of tree where each node stores part of the input data. For example, if one were to input the string 'dog' into an empty trie, the three nodes 'd', 'o', and 'g' would be placed in that order hierarchically, with the 'g' node marked as an endpoint node. If one were to then add the string 'don' into that trie, the only change would be the addition of a 'n' node (marked as an endpoint) as a child of the existing 'o' node - essentially, the path of nodes created by travelling from the root to a marked node forms an input string in the trie. To further this example, adding the string 'do' to the tree would simply mark the existing 'o' node as an endpoint, with no further consumption of space. In this way, for sets of strings that share prefixes, space is used very efficiently as only a single copy of the shared common prefix of the string needs to be stored, rather than storing a copy of that shared prefix for each entry in the set that is stored in the trie. \\ 
\begin{center}
\includegraphics[width=\linewidth,height=0.5\linewidth]{example_trie.tex.tikz}
\\
Trie states after adding the strings 'dog', 'don', and 'do'.\\
Endpoints are darkened.\\
\end{center}
In this implementation of the trie data structure, which takes in strings of bits as input, the tries created can be of four types: bytewise, nibblewise (half byte; four bits), doublebitwise, or bitwise. In each case, the number of bits held at each node in the trie is noted in the name - for example, the nibblewise trie would hold four bits in each node, while the doublebitwise trie would hold two bits in each node. For the tree mentioned above, each node held exactly one character of the inserted string. In the implementation of tries that this project has created, that association of one character per node would be equivalent to bytewise trie, since we are using C strings with one byte per character. \\

There are three operations done against a trie: insert a string into the trie, check whether a string exists in the trie, or delete a string from the trie. In the case of inserting a string of bits into a trie, we first partition the input string into chunks of a length based on the trie. For example, in a bytewise trie, each chunk would be exactly eight bits in size. From those chunks, we walk the trie by going through each chunk and placing it in the trie if needed. For each node n that we walk to, the child node we move to from node n is the one that corresponds to the bits in the current chunk. For example, in a bitwise trie, each node would contain a zero or a one and would have at most two child nodes, one for each possible value that can be represented with a single bit. In a bytewise trie, each node would contain eight bits of information, and at most 256 (2\textsuperscript{8}) children - one for each possible value that can be represented with a byte. \\

For the search and delete operations, we again partition the input string (to be searched for or deleted from the trie) as with the insert case, and proceed to traverse the trie with these chunks. In contrast to the insert case, the search and delete cases allow us to drop out of a trie traversal when a chunk is not found in the trie - the expected chunk not being found in the trie would indicate that the value we are searching for (possibly to delete) does not exist in the trie, and we can return from the operation as appropriate. Although this implementation does not do this, it should be noted that it is also possible to clean up a trie after a delete, by walking back up the trie and pruning any nodes that have no other children. This implementation does not do this cleanup in order to hasten later inserts, which the cleanup might possibly slow down by causing the path to be recreated by the insertion routine. This is a time and space tradeoff, however - the cleanup ``prunes" the trie of sub-tries that contain no endpoints, so that memory use can be reduced.\\

For any operation on a trie (search, insert, delete), the amount of time taken in the worst case scenario is linear with respect to the size of the input string that will be used in the operation. This means that the trie excels when operation timing is strictly limited. For the insert scenario, each chunk of the input must be considered as the trie is traversed (and perhaps extended). In the search scenario (of which the delete scenario is simply a special case), the time taken is bounded above by the size of the input; we can drop out of such a search or delete if part of the expected path in the trie is not present. \\

Additionally, the linear time for all operations is the same across trie node sizes (bytewise, bitwise, etc.). For example, the bytewise trie would reduce the number of iterations over the chunks of input by eight in comparison to the bitwise trie, but this reduction (and the similar reduction that nibblewise and doublebitwise tries would also have) does not change the fact that the time consumed by the operation is still linear with respect to the length of the input string. However, the results show that although this linear time consumption is kept across the different trie node sizes, the actual performance is different.\\
\newpage
\section{Results}
The following charts plot the results of a short experiment comparing the performance of the 4 different trie styles mentioned previously against two sets of strings - one set containing distinct 32-bit strings and the other containing distinct 64-bit strings. In these experiments, the average time spent for each operation is plotted against a subset of one of the two sets previously mentioned. In addition, the plots are further broken down to more easily show any difference in time consumed between when an operation succeeded or when and operation failed (for example, deleting an existing value from a trie would be a ``successful" delete, while attempting to delete a non-existant value from a trie would be a ``failed" delete). Lastly, each graph contains multiple plots to illustrate the significance of the number of bits stored at each node in the trie, which turned out to be the most important variable with respect to time consumed for each operation performed on the trie. \\

\newpage
Since we used unique integers, each insert traversed the same number of nodes for each insert operation. On the other hand, while in the ``success" case, the trie may have needed to be extended to hold parts of the string, this was never an issue for the ``failure" case (since the string to be inserted already existed in the trie). As can be seen in the next four charts, there are a few patterns to the data: most notably, the nibblewise and doublebitwise tries had the best performance, with the nibblewise trie coming out on top. The bytewise and bitwise tries had worse performance than the two trie types already mentioned, but on failed inserts (where the element was already in the trie), the bytewise trie pulled aheaad of the bitwise trie (and started to pull ahead of the doublebitwise trie). However, the opposite was true for the successful insert, where the bytewise trie was the worst performer.\\
\includegraphics[width=\linewidth,height=0.5\linewidth]{32_successful_insert.tex.tikz}
\includegraphics[width=\linewidth,height=0.5\linewidth]{64_successful_insert.tex.tikz}
\includegraphics[width=\linewidth,height=0.5\linewidth]{32_failed_insert.tex.tikz}
\includegraphics[width=\linewidth,height=0.5\linewidth]{64_failed_insert.tex.tikz}

\newpage
On searches, the code is able to kick out of the traversal of the trie when a single character from the target string is not found, so the time consumed on a failed search is bounded from above by the time consumed by the insert function, or by successful searches, which would also have to traverse the entire string's length in the trie. The results on the next four charts concerning searches are similar to the previous results on inserts. Again, the nibblewise and doublebitwise tries had the best performance, with the nibblewise trie winning out. However, unlike in the insert case, the bitwise trie never beat the bytewise trie in any experiment, and in fact was the worst performer overall in the search experiment scenarios.\\
\includegraphics[width=\linewidth,height=0.5\linewidth]{32_successful_search.tex.tikz}
\includegraphics[width=\linewidth,height=0.5\linewidth]{64_successful_search.tex.tikz}
\includegraphics[width=\linewidth,height=0.5\linewidth]{32_failed_search.tex.tikz}
\includegraphics[width=\linewidth,height=0.5\linewidth]{64_failed_search.tex.tikz}

\newpage
Similar to the search case, the delete operation on tries is able to drop out of the operation early when part of a string that is searched for is not found. As one can see in the next four charts, the results are the same as in the case of the searches: the nibblewise and doublebitwise tries performed the best, with the nibblewise trie pulling ahead, and the bitwise trie performed the worst overall. The fact that the search and delete scenarios had very similar results is as expected - since their operations are almost identical, there should not be a difference in the observed performance.\\
\includegraphics[width=\linewidth,height=0.5\linewidth]{32_successful_delete.tex.tikz}
\includegraphics[width=\linewidth,height=0.5\linewidth]{64_successful_delete.tex.tikz}
\includegraphics[width=\linewidth,height=0.5\linewidth]{32_failed_delete.tex.tikz}
\includegraphics[width=\linewidth,height=0.5\linewidth]{64_failed_delete.tex.tikz}

\newpage
From the plots, it is easy to see that although each trie theoretically should perform roughly equivalently (as mentioned previously), the choice of spread causes different behavior in terms of time consumed. Specifically, the trie that considered input string in terms of four bit (nibble-sized) chunks performed the best. This is unexpected - one would think that either extreme (the bitwise or bytewise tries) would perform the best due to the structure of the trie implied by the node size, and the other trie types would gradually perform worse as we increased or decreased the chunk size considered by each node. One possible explanation is that the overhead consumed by the structure around the trie nodes (which was constant across all types) was enough to slow down the bitwise and doublebitwise implementations of the trie, but not enough to slow down the nibblewise trie, which was able to outperform the bytewise trie for other reasons, like those mentioned below.\\

\section{Conclusion and discussion}
In terms of the performance, it's possible that the generated code / CPU cache size was such that it favored the trie whose chunk size was four bits, but that is beyond the scope of this paper. It's also possible that other CPUs would exhibit different behavior (for reference, these data points were generated on a computer built around an AMD FX-6300 CPU, with enough memory such that memory paging out to disk was not an issue during the experiment). This would also explain why the nibblewise trie outperformed the bytewise trie, even though the bytewise trie would have had the least structural overhead of the trie types that were experimented with, due to the breadth over depth approach the bytewise trie would use compared to the depth over breadth approach the bitwise trie uses.\\

One alteration that might be interesting to look at in tries is the use of other trie techniques, like compression of common paths, but this would require adapting those techniques to take into account the different chunk size used by each of the tries in the library above. Compression of common paths would merge common infixes and suffixes as the trie does now with prefixes, with the added computational complexity that would be created to account for the change in the algorithm.\\

Finally, there is also a CPU intrinsic instruction that could be harnessed to further speed actions against the trie up, called Count Leading Zeroes (CLZ). Use of this instruction would allow the algorithm to first examine the input string, find the first set bit, and jump that far down in the trie, rather than having to manually walk through the path of zeroes. This approach would keep track of paths in the trie composed entirely of leading zeroes to facilitate this purpose, which would only require storage size linear with respect to the number of bits stored in the longest string in the trie. \\
%TODO mention that using a custom allocator (bins holding blocks of the size of the node) might speed this up even more.
\newpage
%\ttdefault
\section{Manual Page}
%TODO make this in monospace, like a real man page!
\noindent NAME

make\_trie, trie\_insert, trie\_delete, trie\_search - Create and interact with tries.\\

\noindent SYNOPSIS

Trie *make\_trie(enum trie\_type type);\\
\indent int trie\_insert(Trie *t, void *item, int sz);\\
\indent int trie\_delete(Trie *t, void *item, int sz);\\
\indent int trie\_search(Trie *t, void *item, int sz);\\

\noindent DESCRIPTION

\noindent The make\_trie() function creates a trie struct in memory and returns a pointer to it. The type of trie (the branching factor) must be specified as one of the following enumerated values: BIT, DBLBIT, NIBBLE, BYTE.\\
The trie\_insert() function takes the data of length sz bytes which is pointed to by item and attempts to insert it into the trie t.\\
The trie\_delete() function takes the data of length sz bytes which is pointed to by item and attempts to remove it from the trie t.\\
The trie\_search() function takes the data of length sz bytes which is pointed to by item and attempts to find it in the trie t.\\

\noindent RETURN VALUE

\noindent The return value of make\_trie() is a pointer to a struct of type Trie which is suitable for use with the trie\_insert, trie\_delete, and trie\_search functions.\\
The trie\_insert function returns 1 when the specified value is inserted into the given trie and 0 when the specified value is already in the tree.\\
The trie\_delete function returns 1 when the specified value is removed from the given trie and 0 when the specified value is not already in the tree to remove.\\
The trie\_search function returns 1 when the specified value is found in the given trie and 0 if the specified value is not found.\\

\newpage
\section{Code}
\lstinputlisting[label=header, caption=trie.h, language=C, breaklines=true]{trie.h}
\newpage
\lstinputlisting[label=code, caption=trie.c, language=C, breaklines=true]{trie.c}

\end{document}
